Разработать подпрограммы для воспроизведения
на ЭВМ модели, определяемой вариантом задания. Требуется:
\begin{enumerate}
  \item Построить дискретную модель. Способ построения
  дискретной модели определяется вариантом задания. Обеспечить
  2-ой порядок аппроксимации уравнения и граничных условий модели
  на решении. Получить выражение для главного члена погрешности
  аппроксимации уравнения и граничных условий.

  \item Для решения алгебраической системы использовать метод,
  определяемый вариантом задания.

  \item Использовать модульный принцип программирования. Вычисление
  коэффициентов и параметров модели, а также решение алгебраической
  системы, написать в виде подпрограмм или подпрограмм-функций. Снабдить
  программы комментариями.

  \item Разработать не менее двух тестов для проверки работоспособности
  подпрограмм, реализующих модель. Тесты получить на основе метода пробных
  решений и анализа главного члена погрешности. Один из тестов должен
  иметь решение, на котором погрешность аппроксимации стационарного уравнения
  и граничных условий равна нулю, второй тест - с ненулевой погрешностью
  аппроксимации. Разработать специальные тесты для проверки подпрограмм решения
  алгебраической системы.

  \item Получить для каждого теста зависимость нормы погрешности решения
  от величины шага дискретности. Проанализировать порядки теоретической и
  экспериментально полученной погрешности. При решении тестов для нестационарных
  задач получить стационарные решения.

  \item Применить разработанную подпрограмму для решения конкретной модели
  с параметрами, указанными преподавателем. Получить экспериментально времена
  выполнения программ в зависимости от числа разбиений.
\end{enumerate}
