Для проверки построенной модели используем следующий набор пробных решений.
\begin{table}[H]
    \centering
    \begin{tabular}{*4c}
        \toprule
        \makecell{№\\теста} & $k_1(x, y)$ & $k_2(x, y)$ & $u(x, y)$ \\
        \midrule
        1 & $(x + y + 1)^2$ & $(x + y + 1)^3$ & $1$ \\
        2 & $x + y + 1$ & $x + y + 1$ & $2 x + 3 y + 1$ \\
        3 & $(x + y + 1)^2$ & $(x + y + 1)^2$ & $2 x + 3 y + 1$ \\
        4 & $(x + y + 1)^3$ & $(x + y + 1)^3$ & $2 x + 3 y + 1$ \\
        5 & $1$ & $1$ & $(2 x + 3 y + 1)^2$ \\
        6 & $x + y + 1$ & $x + y + 1$ & $(2 x + 3 y + 1)^2$ \\
        7 & $x + y + 1$ & $x + y + 1$ & $(2 x + 3 y + 1)^3$ \\
        8 & $\cos(x y) + 2$ & $\sin(x)\sin(y) + 2$ & $x \ln(y + 1)$ \\
        \bottomrule
    \end{tabular}
\end{table}
Для всех тестов определим следующие значения
\[ a = 0,\quad b = 1 \]
\[ c = 0,\quad d = 2 \]
\[ \chi_1 = 1,\quad \chi_3 = 3,\quad \chi_4 = 4 \]
Функция $f$, как и функции $g_1$, $g_2$, $g_3$, $g_4$, вычисляется программно с
использованием символьных вычислений и не приведена в описании теста.

Погрешность решения задачи будем вычислять тремя разными способами.
\begin{align*}
    \delta_1 &= \frac{\norm{\vec{x} - \tilde{\vec{x}}}_1}{\norm{\vec{x}}_1} &
    \delta_2 &= \frac{\norm{\vec{x} - \tilde{\vec{x}}}_2}{\norm{\vec{x}}_2} &
    \delta_3 &= \frac{\norm{\vec{x} - \tilde{\vec{x}}}_{\infty}}{\norm{\vec{x}}_{\infty}} \\
\end{align*}
Значение $\varepsilon$ для метода сопряженных градиентов равно $10^{-8}$.

\newpage
Тест №1 -- константный. Присутствует только ошибка округления.
\begin{table}[H]
	\centering
	\begin{tabular}{*6c}
		\toprule
		$N_x$ & $N_y$ & $n$ & $\delta_1$ & $\delta_2$ & $\delta_3$ \\
		\midrule
		2 & 2 & 7 & 1.22e-15 & 2.35e-15 & 6.44e-15 \\
		4 & 2 & 13 & 2.34e-11 & 4.84e-11 & 1.26e-10 \\
		2 & 4 & 12 & 1.56e-14 & 3.11e-14 & 9.48e-14 \\
		4 & 4 & 23 & 2.11e-12 & 3.95e-12 & 1.24e-11 \\
		8 & 4 & 47 & 2.40e-10 & 4.02e-10 & 1.28e-09 \\
		4 & 8 & 44 & 5.50e-11 & 1.09e-10 & 3.91e-10 \\
		8 & 8 & 72 & 2.10e-09 & 2.89e-09 & 1.00e-08 \\
		16 & 8 & 124 & 8.10e-09 & 1.07e-08 & 4.34e-08 \\
		8 & 16 & 118 & 1.56e-09 & 2.17e-09 & 9.56e-09 \\
		16 & 16 & 151 & 2.80e-09 & 4.10e-09 & 2.15e-08 \\
		32 & 16 & 247 & 8.38e-09 & 1.23e-08 & 9.07e-08 \\
		16 & 32 & 235 & 2.67e-09 & 4.65e-09 & 5.82e-08 \\
		32 & 32 & 294 & 7.84e-09 & 1.01e-08 & 3.86e-08 \\
		\bottomrule
	\end{tabular}
	\caption{Тест №1 (явный метод)}
\end{table}
\begin{table}[H]
	\centering
	\begin{tabular}{*6c}
		\toprule
		$N_x$ & $N_y$ & $n$ & $\delta_1$ & $\delta_2$ & $\delta_3$ \\
		\midrule
		2 & 2 & 5 & 8.64e-17 & 1.33e-16 & 3.33e-16 \\
		4 & 2 & 7 & 4.53e-10 & 8.68e-10 & 3.13e-09 \\
		2 & 4 & 7 & 7.49e-10 & 9.93e-10 & 2.09e-09 \\
		4 & 4 & 9 & 1.99e-09 & 2.74e-09 & 7.46e-09 \\
		8 & 4 & 11 & 3.87e-09 & 5.31e-09 & 1.67e-08 \\
		4 & 8 & 12 & 1.28e-10 & 1.70e-10 & 5.20e-10 \\
		8 & 8 & 15 & 1.02e-09 & 1.59e-09 & 5.04e-09 \\
		16 & 8 & 19 & 1.12e-09 & 1.65e-09 & 6.57e-09 \\
		8 & 16 & 20 & 4.73e-10 & 5.98e-10 & 1.70e-09 \\
		16 & 16 & 26 & 1.31e-09 & 2.05e-09 & 9.27e-09 \\
		32 & 16 & 33 & 2.90e-09 & 4.13e-09 & 1.76e-08 \\
		16 & 32 & 35 & 2.26e-09 & 3.34e-09 & 1.27e-08 \\
		32 & 32 & 47 & 6.73e-09 & 9.72e-09 & 3.67e-08 \\
		\bottomrule
	\end{tabular}
	\caption{Тест №1 (неявный метод)}
\end{table}

\newpage
Тест №2 -- линейный. Присутствует только ошибка округления.
\begin{table}[H]
	\centering
	\begin{tabular}{*6c}
		\toprule
		$N_x$ & $N_y$ & $n$ & $\delta_1$ & $\delta_2$ & $\delta_3$ \\
		\midrule
		2 & 2 & 7 & 4.05e-16 & 6.19e-16 & 8.88e-16 \\
		4 & 2 & 13 & 2.71e-11 & 4.68e-11 & 7.66e-11 \\
		2 & 4 & 11 & 3.08e-16 & 4.40e-16 & 6.91e-16 \\
		4 & 4 & 21 & 1.15e-10 & 2.03e-10 & 3.97e-10 \\
		8 & 4 & 43 & 6.25e-09 & 7.26e-09 & 1.01e-08 \\
		4 & 8 & 34 & 7.81e-10 & 1.09e-09 & 1.92e-09 \\
		8 & 8 & 53 & 7.59e-09 & 9.49e-09 & 2.11e-08 \\
		16 & 8 & 99 & 4.65e-09 & 5.88e-09 & 9.04e-09 \\
		8 & 16 & 65 & 1.88e-09 & 2.45e-09 & 6.74e-09 \\
		16 & 16 & 105 & 3.67e-09 & 4.57e-09 & 1.05e-08 \\
		32 & 16 & 196 & 1.11e-08 & 1.47e-08 & 3.09e-08 \\
		16 & 32 & 131 & 3.73e-09 & 4.28e-09 & 6.71e-09 \\
		32 & 32 & 208 & 8.67e-09 & 9.88e-09 & 2.30e-08 \\
		\bottomrule
	\end{tabular}
	\caption{Тест №2 (явный метод)}
\end{table}
\begin{table}[H]
	\centering
	\begin{tabular}{*6c}
		\toprule
		$N_x$ & $N_y$ & $n$ & $\delta_1$ & $\delta_2$ & $\delta_3$ \\
		\midrule
		2 & 2 & 5 & 5.43e-17 & 1.09e-16 & 1.97e-16 \\
		4 & 2 & 6 & 3.86e-10 & 4.45e-10 & 5.65e-10 \\
		2 & 4 & 7 & 7.82e-11 & 1.11e-10 & 1.64e-10 \\
		4 & 4 & 8 & 3.94e-09 & 4.93e-09 & 8.23e-09 \\
		8 & 4 & 10 & 4.51e-10 & 5.48e-10 & 9.80e-10 \\
		4 & 8 & 11 & 3.19e-10 & 3.91e-10 & 5.13e-10 \\
		8 & 8 & 14 & 1.39e-09 & 1.70e-09 & 3.02e-09 \\
		16 & 8 & 16 & 3.16e-09 & 3.94e-09 & 7.61e-09 \\
		8 & 16 & 18 & 1.04e-09 & 1.21e-09 & 1.76e-09 \\
		16 & 16 & 24 & 2.92e-09 & 3.68e-09 & 7.74e-09 \\
		32 & 16 & 29 & 4.89e-09 & 6.02e-09 & 1.03e-08 \\
		16 & 32 & 32 & 7.51e-09 & 9.51e-09 & 1.74e-08 \\
		32 & 32 & 44 & 6.04e-09 & 7.63e-09 & 1.49e-08 \\
		\bottomrule
	\end{tabular}
	\caption{Тест №2 (неявный метод)}
\end{table}

\newpage
Тест №3 -- линейный. Присутствует как ошибка округления, так и ошибка аппроксимации.
\begin{table}[H]
	\centering
	\begin{tabular}{*6c}
		\toprule
		$N_x$ & $N_y$ & $n$ & $\delta_1$ & $\delta_2$ & $\delta_3$ \\
		\midrule
		2 & 2 & 7 & 5.37e-03 & 7.88e-03 & 1.21e-02 \\
		4 & 2 & 13 & 5.02e-03 & 6.93e-03 & 9.76e-03 \\
		2 & 4 & 11 & 1.93e-03 & 2.96e-03 & 5.12e-03 \\
		4 & 4 & 22 & 1.27e-03 & 1.88e-03 & 3.38e-03 \\
		8 & 4 & 48 & 1.20e-03 & 1.71e-03 & 2.89e-03 \\
		4 & 8 & 39 & 4.65e-04 & 6.77e-04 & 1.33e-03 \\
		8 & 8 & 67 & 2.89e-04 & 4.26e-04 & 8.80e-04 \\
		16 & 8 & 132 & 2.70e-04 & 3.87e-04 & 7.70e-04 \\
		8 & 16 & 88 & 1.12e-04 & 1.59e-04 & 3.37e-04 \\
		16 & 16 & 138 & 6.74e-05 & 9.82e-05 & 2.23e-04 \\
		32 & 16 & 258 & 6.23e-05 & 8.87e-05 & 1.97e-04 \\
		16 & 32 & 168 & 2.73e-05 & 3.82e-05 & 8.45e-05 \\
		32 & 32 & 273 & 1.62e-05 & 2.34e-05 & 5.61e-05 \\
		\bottomrule
	\end{tabular}
	\caption{Тест №3 (явный метод)}
\end{table}
\begin{table}[H]
	\centering
	\begin{tabular}{*6c}
		\toprule
		$N_x$ & $N_y$ & $n$ & $\delta_1$ & $\delta_2$ & $\delta_3$ \\
		\midrule
		2 & 2 & 5 & 5.37e-03 & 7.88e-03 & 1.21e-02 \\
		4 & 2 & 6 & 5.02e-03 & 6.93e-03 & 9.76e-03 \\
		2 & 4 & 7 & 1.93e-03 & 2.96e-03 & 5.12e-03 \\
		4 & 4 & 9 & 1.27e-03 & 1.88e-03 & 3.38e-03 \\
		8 & 4 & 10 & 1.20e-03 & 1.71e-03 & 2.89e-03 \\
		4 & 8 & 11 & 4.65e-04 & 6.77e-04 & 1.33e-03 \\
		8 & 8 & 14 & 2.89e-04 & 4.26e-04 & 8.80e-04 \\
		16 & 8 & 17 & 2.70e-04 & 3.87e-04 & 7.70e-04 \\
		8 & 16 & 18 & 1.12e-04 & 1.59e-04 & 3.37e-04 \\
		16 & 16 & 24 & 6.74e-05 & 9.82e-05 & 2.23e-04 \\
		32 & 16 & 29 & 6.23e-05 & 8.87e-05 & 1.97e-04 \\
		16 & 32 & 32 & 2.73e-05 & 3.82e-05 & 8.45e-05 \\
		32 & 32 & 44 & 1.62e-05 & 2.34e-05 & 5.61e-05 \\
		\bottomrule
	\end{tabular}
	\caption{Тест №3 (неявный метод)}
\end{table}

\newpage
Тест №4 -- линейный. Присутствует как ошибка округления, так и ошибка аппроксимации.
\begin{table}[H]
	\centering
	\begin{tabular}{*6c}
		\toprule
		$N_x$ & $N_y$ & $n$ & $\delta_1$ & $\delta_2$ & $\delta_3$ \\
		\midrule
		2 & 2 & 7 & 1.60e-02 & 2.34e-02 & 3.57e-02 \\
		4 & 2 & 15 & 1.53e-02 & 2.11e-02 & 2.98e-02 \\
		2 & 4 & 11 & 5.66e-03 & 8.71e-03 & 1.51e-02 \\
		4 & 4 & 24 & 3.90e-03 & 5.73e-03 & 1.02e-02 \\
		8 & 4 & 53 & 3.73e-03 & 5.28e-03 & 8.86e-03 \\
		4 & 8 & 44 & 1.36e-03 & 2.01e-03 & 4.01e-03 \\
		8 & 8 & 87 & 8.94e-04 & 1.31e-03 & 2.68e-03 \\
		16 & 8 & 167 & 8.48e-04 & 1.20e-03 & 2.36e-03 \\
		8 & 16 & 120 & 3.25e-04 & 4.70e-04 & 1.02e-03 \\
		16 & 16 & 195 & 2.09e-04 & 3.01e-04 & 6.80e-04 \\
		32 & 16 & 360 & 1.96e-04 & 2.76e-04 & 6.04e-04 \\
		16 & 32 & 241 & 7.90e-05 & 1.13e-04 & 2.56e-04 \\
		32 & 32 & 380 & 5.00e-05 & 7.16e-05 & 1.71e-04 \\
		\bottomrule
	\end{tabular}
	\caption{Тест №4 (явный метод)}
\end{table}
\begin{table}[H]
	\centering
	\begin{tabular}{*6c}
		\toprule
		$N_x$ & $N_y$ & $n$ & $\delta_1$ & $\delta_2$ & $\delta_3$ \\
		\midrule
		2 & 2 & 5 & 1.60e-02 & 2.34e-02 & 3.57e-02 \\
		4 & 2 & 6 & 1.53e-02 & 2.11e-02 & 2.98e-02 \\
		2 & 4 & 7 & 5.66e-03 & 8.71e-03 & 1.51e-02 \\
		4 & 4 & 9 & 3.90e-03 & 5.73e-03 & 1.02e-02 \\
		8 & 4 & 10 & 3.73e-03 & 5.28e-03 & 8.86e-03 \\
		4 & 8 & 11 & 1.36e-03 & 2.01e-03 & 4.01e-03 \\
		8 & 8 & 14 & 8.94e-04 & 1.31e-03 & 2.68e-03 \\
		16 & 8 & 16 & 8.48e-04 & 1.20e-03 & 2.36e-03 \\
		8 & 16 & 18 & 3.25e-04 & 4.70e-04 & 1.02e-03 \\
		16 & 16 & 24 & 2.09e-04 & 3.01e-04 & 6.80e-04 \\
		32 & 16 & 29 & 1.96e-04 & 2.76e-04 & 6.04e-04 \\
		16 & 32 & 31 & 7.90e-05 & 1.13e-04 & 2.56e-04 \\
		32 & 32 & 43 & 5.00e-05 & 7.16e-05 & 1.71e-04 \\
		\bottomrule
	\end{tabular}
	\caption{Тест №4 (неявный метод)}
\end{table}

\newpage
Тест №5 -- полиномиальный. Присутствует только ошибка округления.
\begin{table}[H]
	\centering
	\begin{tabular}{*6c}
		\toprule
		$N_x$ & $N_y$ & $n$ & $\delta_1$ & $\delta_2$ & $\delta_3$ \\
		\midrule
		2 & 2 & 7 & 8.80e-17 & 1.26e-16 & 1.75e-16 \\
		4 & 2 & 13 & 1.26e-13 & 1.85e-13 & 2.35e-13 \\
		2 & 4 & 11 & 2.21e-16 & 2.30e-16 & 1.75e-16 \\
		4 & 4 & 21 & 1.02e-14 & 1.17e-14 & 1.24e-14 \\
		8 & 4 & 37 & 3.46e-09 & 3.57e-09 & 4.31e-09 \\
		4 & 8 & 26 & 4.78e-09 & 5.54e-09 & 8.19e-09 \\
		8 & 8 & 43 & 4.88e-09 & 5.06e-09 & 6.56e-09 \\
		16 & 8 & 81 & 6.43e-09 & 6.76e-09 & 1.05e-08 \\
		8 & 16 & 54 & 2.22e-09 & 2.71e-09 & 7.02e-09 \\
		16 & 16 & 83 & 8.25e-09 & 8.74e-09 & 1.19e-08 \\
		32 & 16 & 146 & 9.20e-09 & 1.01e-08 & 1.36e-08 \\
		16 & 32 & 105 & 4.99e-09 & 6.05e-09 & 9.70e-09 \\
		32 & 32 & 163 & 9.63e-09 & 1.12e-08 & 1.92e-08 \\
		\bottomrule
	\end{tabular}
	\caption{Тест №5 (явный метод)}
\end{table}
\begin{table}[H]
	\centering
	\begin{tabular}{*6c}
		\toprule
		$N_x$ & $N_y$ & $n$ & $\delta_1$ & $\delta_2$ & $\delta_3$ \\
		\midrule
		2 & 2 & 5 & 1.41e-16 & 1.53e-16 & 1.75e-16 \\
		4 & 2 & 6 & 1.81e-10 & 1.80e-10 & 2.31e-10 \\
		2 & 4 & 6 & 1.10e-09 & 1.37e-09 & 1.87e-09 \\
		4 & 4 & 8 & 2.60e-09 & 2.58e-09 & 2.48e-09 \\
		8 & 4 & 9 & 3.87e-09 & 3.70e-09 & 3.94e-09 \\
		4 & 8 & 10 & 9.67e-10 & 1.07e-09 & 1.16e-09 \\
		8 & 8 & 13 & 4.84e-09 & 5.13e-09 & 6.05e-09 \\
		16 & 8 & 15 & 6.37e-09 & 6.34e-09 & 6.52e-09 \\
		8 & 16 & 17 & 1.82e-09 & 1.95e-09 & 2.46e-09 \\
		16 & 16 & 23 & 7.17e-09 & 7.69e-09 & 1.08e-08 \\
		32 & 16 & 28 & 8.27e-09 & 8.44e-09 & 1.12e-08 \\
		16 & 32 & 31 & 8.83e-09 & 9.43e-09 & 1.10e-08 \\
		32 & 32 & 43 & 5.85e-09 & 6.33e-09 & 8.54e-09 \\
		\bottomrule
	\end{tabular}
	\caption{Тест №5 (неявный метод)}
\end{table}

\newpage
Тест №6 -- полиномиальный. Присутствует как ошибка округления, так и ошибка аппроксимации.
\begin{table}[H]
	\centering
	\begin{tabular}{*6c}
		\toprule
		$N_x$ & $N_y$ & $n$ & $\delta_1$ & $\delta_2$ & $\delta_3$ \\
		\midrule
		2 & 2 & 7 & 7.61e-03 & 8.62e-03 & 9.27e-03 \\
		4 & 2 & 13 & 7.80e-03 & 8.50e-03 & 8.32e-03 \\
		2 & 4 & 11 & 2.50e-03 & 2.99e-03 & 3.54e-03 \\
		4 & 4 & 21 & 1.98e-03 & 2.27e-03 & 2.55e-03 \\
		8 & 4 & 44 & 1.98e-03 & 2.22e-03 & 2.38e-03 \\
		4 & 8 & 33 & 6.35e-04 & 7.40e-04 & 9.09e-04 \\
		8 & 8 & 55 & 4.71e-04 & 5.45e-04 & 6.63e-04 \\
		16 & 8 & 99 & 4.64e-04 & 5.27e-04 & 6.17e-04 \\
		8 & 16 & 65 & 1.57e-04 & 1.81e-04 & 2.29e-04 \\
		16 & 16 & 105 & 1.13e-04 & 1.30e-04 & 1.68e-04 \\
		32 & 16 & 196 & 1.10e-04 & 1.25e-04 & 1.56e-04 \\
		16 & 32 & 131 & 3.90e-05 & 4.46e-05 & 5.73e-05 \\
		32 & 32 & 208 & 2.74e-05 & 3.16e-05 & 4.21e-05 \\
		\bottomrule
	\end{tabular}
	\caption{Тест №6 (явный метод)}
\end{table}
\begin{table}[H]
	\centering
	\begin{tabular}{*6c}
		\toprule
		$N_x$ & $N_y$ & $n$ & $\delta_1$ & $\delta_2$ & $\delta_3$ \\
		\midrule
		2 & 2 & 5 & 7.61e-03 & 8.62e-03 & 9.27e-03 \\
		4 & 2 & 6 & 7.80e-03 & 8.50e-03 & 8.32e-03 \\
		2 & 4 & 7 & 2.50e-03 & 2.99e-03 & 3.54e-03 \\
		4 & 4 & 8 & 1.98e-03 & 2.27e-03 & 2.55e-03 \\
		8 & 4 & 10 & 1.98e-03 & 2.22e-03 & 2.38e-03 \\
		4 & 8 & 11 & 6.35e-04 & 7.40e-04 & 9.09e-04 \\
		8 & 8 & 14 & 4.71e-04 & 5.45e-04 & 6.63e-04 \\
		16 & 8 & 16 & 4.64e-04 & 5.27e-04 & 6.17e-04 \\
		8 & 16 & 18 & 1.57e-04 & 1.81e-04 & 2.29e-04 \\
		16 & 16 & 24 & 1.13e-04 & 1.30e-04 & 1.68e-04 \\
		32 & 16 & 29 & 1.10e-04 & 1.25e-04 & 1.56e-04 \\
		16 & 32 & 32 & 3.90e-05 & 4.46e-05 & 5.73e-05 \\
		32 & 32 & 44 & 2.74e-05 & 3.16e-05 & 4.21e-05 \\
		\bottomrule
	\end{tabular}
	\caption{Тест №6 (неявный метод)}
\end{table}

\newpage
Тест №7 -- полиномиальный. Присутствует как ошибка округления, так и ошибка аппроксимации.
\begin{table}[H]
	\centering
	\begin{tabular}{*6c}
		\toprule
		$N_x$ & $N_y$ & $n$ & $\delta_1$ & $\delta_2$ & $\delta_3$ \\
		\midrule
		2 & 2 & 7 & 2.51e-02 & 2.39e-02 & 1.84e-02 \\
		4 & 2 & 13 & 2.81e-02 & 2.68e-02 & 2.20e-02 \\
		2 & 4 & 11 & 7.74e-03 & 7.27e-03 & 4.96e-03 \\
		4 & 4 & 21 & 7.43e-03 & 7.09e-03 & 6.00e-03 \\
		8 & 4 & 44 & 7.81e-03 & 7.52e-03 & 6.43e-03 \\
		4 & 8 & 33 & 2.15e-03 & 1.99e-03 & 1.29e-03 \\
		8 & 8 & 55 & 1.88e-03 & 1.81e-03 & 1.57e-03 \\
		16 & 8 & 99 & 1.93e-03 & 1.88e-03 & 1.67e-03 \\
		8 & 16 & 64 & 5.57e-04 & 5.09e-04 & 3.24e-04 \\
		16 & 16 & 105 & 4.66e-04 & 4.49e-04 & 3.96e-04 \\
		32 & 16 & 196 & 4.70e-04 & 4.60e-04 & 4.20e-04 \\
		16 & 32 & 131 & 1.41e-04 & 1.28e-04 & 8.13e-05 \\
		32 & 32 & 207 & 1.15e-04 & 1.11e-04 & 9.93e-05 \\
		\bottomrule
	\end{tabular}
	\caption{Тест №7 (явный метод)}
\end{table}
\begin{table}[H]
	\centering
	\begin{tabular}{*6c}
		\toprule
		$N_x$ & $N_y$ & $n$ & $\delta_1$ & $\delta_2$ & $\delta_3$ \\
		\midrule
		2 & 2 & 4 & 2.51e-02 & 2.39e-02 & 1.84e-02 \\
		4 & 2 & 6 & 2.81e-02 & 2.68e-02 & 2.20e-02 \\
		2 & 4 & 6 & 7.74e-03 & 7.27e-03 & 4.96e-03 \\
		4 & 4 & 8 & 7.43e-03 & 7.09e-03 & 6.00e-03 \\
		8 & 4 & 10 & 7.81e-03 & 7.52e-03 & 6.43e-03 \\
		4 & 8 & 11 & 2.15e-03 & 1.99e-03 & 1.29e-03 \\
		8 & 8 & 14 & 1.88e-03 & 1.81e-03 & 1.57e-03 \\
		16 & 8 & 17 & 1.93e-03 & 1.88e-03 & 1.67e-03 \\
		8 & 16 & 18 & 5.57e-04 & 5.09e-04 & 3.24e-04 \\
		16 & 16 & 24 & 4.66e-04 & 4.49e-04 & 3.96e-04 \\
		32 & 16 & 30 & 4.70e-04 & 4.60e-04 & 4.20e-04 \\
		16 & 32 & 32 & 1.41e-04 & 1.28e-04 & 8.13e-05 \\
		32 & 32 & 45 & 1.15e-04 & 1.11e-04 & 9.93e-05 \\
		\bottomrule
	\end{tabular}
	\caption{Тест №7 (неявный метод)}
\end{table}

\newpage
Тест №8 добавлен для дополнительной проверки написанной модели.
Присутствует как ошибка округления, так и ошибка аппроксимации.
\begin{table}[H]
	\centering
	\begin{tabular}{*6c}
		\toprule
		$N_x$ & $N_y$ & $n$ & $\delta_1$ & $\delta_2$ & $\delta_3$ \\
		\midrule
		2 & 2 & 7 & 1.79e-02 & 1.57e-02 & 1.58e-02 \\
		4 & 2 & 13 & 1.31e-02 & 1.26e-02 & 1.10e-02 \\
		2 & 4 & 11 & 1.07e-02 & 1.12e-02 & 1.24e-02 \\
		4 & 4 & 21 & 4.37e-03 & 4.40e-03 & 4.07e-03 \\
		8 & 4 & 43 & 3.28e-03 & 3.92e-03 & 4.27e-03 \\
		4 & 8 & 31 & 2.59e-03 & 2.75e-03 & 3.24e-03 \\
		8 & 8 & 51 & 1.06e-03 & 1.11e-03 & 1.29e-03 \\
		16 & 8 & 95 & 7.41e-04 & 9.68e-04 & 1.32e-03 \\
		8 & 16 & 58 & 6.30e-04 & 6.74e-04 & 8.28e-04 \\
		16 & 16 & 94 & 2.52e-04 & 2.66e-04 & 3.52e-04 \\
		32 & 16 & 173 & 1.69e-04 & 2.23e-04 & 3.59e-04 \\
		16 & 32 & 113 & 1.55e-04 & 1.66e-04 & 2.07e-04 \\
		32 & 32 & 180 & 6.05e-05 & 6.40e-05 & 9.02e-05 \\
		\bottomrule
	\end{tabular}
	\caption{Тест №8 (явный метод)}
\end{table}
\begin{table}[H]
	\centering
	\begin{tabular}{*6c}
		\toprule
		$N_x$ & $N_y$ & $n$ & $\delta_1$ & $\delta_2$ & $\delta_3$ \\
		\midrule
		2 & 2 & 5 & 1.79e-02 & 1.57e-02 & 1.58e-02 \\
		4 & 2 & 6 & 1.31e-02 & 1.26e-02 & 1.10e-02 \\
		2 & 4 & 7 & 1.07e-02 & 1.12e-02 & 1.24e-02 \\
		4 & 4 & 9 & 4.37e-03 & 4.40e-03 & 4.07e-03 \\
		8 & 4 & 10 & 3.28e-03 & 3.92e-03 & 4.27e-03 \\
		4 & 8 & 11 & 2.59e-03 & 2.75e-03 & 3.24e-03 \\
		8 & 8 & 14 & 1.06e-03 & 1.11e-03 & 1.29e-03 \\
		16 & 8 & 16 & 7.41e-04 & 9.68e-04 & 1.32e-03 \\
		8 & 16 & 18 & 6.30e-04 & 6.74e-04 & 8.28e-04 \\
		16 & 16 & 24 & 2.52e-04 & 2.66e-04 & 3.52e-04 \\
		32 & 16 & 28 & 1.69e-04 & 2.23e-04 & 3.59e-04 \\
		16 & 32 & 33 & 1.55e-04 & 1.66e-04 & 2.07e-04 \\
		32 & 32 & 44 & 6.05e-05 & 6.40e-05 & 9.02e-05 \\
		\bottomrule
	\end{tabular}
	\caption{Тест №8 (неявный метод)}
\end{table}

\newpage
Дадим общую информацию по всем тестам.
\begin{itemize}
	\item Неявный метод показывает гораздо большую скорость
	сходимости.
	\item При наличии ошибки аппроксимации с увеличением числа
	разбиений в \textit{два} раза погрешность уменьшается примерно в
	\textit{четыре} раза: значение может быть болеше или меньше в зависимости от
	способа вычисления погрешности, но приближается к четырем с
	увеличением числа разбиений до достаточных величин.
	\item Напомним, что одномерная нуменация неизвестных велась
	вдоль оси $x$, поэтому в случае, когда число неизвестных
	вдоль выбранного направления больше, то есть $N_x > N_y$,
	можем наблюдать увеличение числа итераций метода, необходимых
	для получения решения.
\end{itemize}