% done
В прямоугольнике $[a, b] \times [c, d]$ введем сетку с равномерным шагом $h_x$ и $h_y$.
\[ h_x = \frac{b - a}{N_x} \]
\[ h_y = \frac{d - c}{N_y} \]
где $N_x$ и $N_y$ -- число разбиений сетки. Тогда координаты узлов построенной сетки
вычисляются следуюим образом
\[ x_i = a + i h_x,\quad 0 \leq i \leq N_x \]
\[ y_j = c + j h_y,\quad 0 \leq j \leq N_y \]
Также введем вспомогательную сетку, координаты узлов которой вычисляются как
\[ x_{i+1/2} = \frac{x_{i+1} - x_i}{2},\quad 0 \leq i < N_x \]
\[ y_{j+1/2} = \frac{y_{j+1} - y_j}{2},\quad 0 \leq j < N_y \]
Теперь построим разностную схему в узлах основной сетки.

% done
\subsection{Аппроксимация во внутрениих точках}
Проинтегрируем уравнение (\ref{main-equation}) во внутренних точках сетки.
\[
  - \left[
  \underbrace{ \Int{x_{i-1/2}}{x_{i+1/2}}\Int{y_{j-1/2}}{y_{j+1/2}} \Inner{\pdv{x}\left( k_1 \pdv{u}{x} \right)} \Ddd{x}{y} }_{I_1} +
  \underbrace{ \Int{x_{i-1/2}}{x_{i+1/2}}\Int{y_{j-1/2}}{y_{j+1/2}} \Inner{\pdv{y}\left( k_2 \pdv{u}{y} \right)} \Ddd{x}{y} }_{I_2}
  \right] =
  \underbrace{ \Int{x_{i-1/2}}{x_{i+1/2}}\Int{y_{j-1/2}}{y_{j+1/2}} \Inner{f} \Ddd{x}{y} }_{I_3}
\]
Проинтегрируем левую часть один раз.
\[ I_1 = \Int{y_{j-1/2}}{y_{j+1/2}} \Inner{k_1(x_{i+1/2},y) \pdv{u}{x} \MyVert{x = x_{i+1/2}}} \Dd{y} - \Int{y_{j-1/2}}{y_{j+1/2}} \Inner{k_1(x_{i-1/2},y) \pdv{u}{x} \MyVert{x = x_{i-1/2}}} \Dd{y} \]
\[ I_2 = \Int{x_{i-1/2}}{x_{i+1/2}} \Inner{k_2(x,y_{j+1/2}) \pdv{u}{y} \MyVert{y = y_{j+1/2}}} \Dd{x} - \Int{x_{i-1/2}}{x_{i+1/2}} \Inner{k_2(x,y_{j-1/2}) \pdv{u}{y} \MyVert{y = y_{j-1/2}}} \Dd{x} \]
Найдем интегралы с помощью формулы средних прямоугольников.
\[ I_1 \approx h_y k_1(x_{i+1/2},y_{j}) \pdv{u}{x} \MyVvert{x = x_{i+1/2}}{y = y_{j}} - h_y k_1(x_{i-1/2},y_{j}) \pdv{u}{x} \MyVvert{x = x_{i-1/2}}{y = y_{j}} \]
\[ I_2 \approx h_x k_2(x_{i},y_{j+1/2}) \pdv{u}{y} \MyVvert{x = x_{i}}{y = y_{j+1/2}} - h_x k_2(x_{i},y_{j-1/2}) \pdv{u}{y} \MyVvert{x = x_{i}}{y = y_{j-1/2}} \]
Теперь воспользуемся формулами численного дифференцирования.
\[ I_1 \approx h_y k_1(x_{i+1/2},y_{j}) \frac{v_{i+1,j} - v_{i,j}}{h_x} - h_y k_1(x_{i-1/2},y_{j}) \frac{v_{i,j} - v_{i-1,j}}{h_x} \]
\[ I_2 \approx h_x k_2(x_{i},y_{j+1/2}) \frac{v_{i,j+1} - v_{i,j}}{h_y} - h_x k_2(x_{i},y_{j-1/2}) \frac{v_{i,j} - v_{i,j-1}}{h_y} \]
Также с помощью формулы средних прямоугольников найдем интеграл в правой части.
\[ I_3 \approx h_x h_y f_{i,j} \]
Получим уравнение разностной схемы для $i = \overline{1,N_x-1}$ и $j = \overline{1,N_y-1}$.
\begin{multline*}
  - \left[
  h_y k_1(x_{i+1/2},y_{j}) \frac{v_{i+1,j} - v_{i,j}}{h_x} - h_y k_1(x_{i-1/2},y_{j}) \frac{v_{i,j} - v_{i-1,j}}{h_x} + \right. \\
  \left. +
  h_x k_2(x_{i},y_{j+1/2}) \frac{v_{i,j+1} - v_{i,j}}{h_y} - h_x k_2(x_{i},y_{j-1/2}) \frac{v_{i,j} - v_{i,j-1}}{h_y}
  \right] =
  h_x h_y f_{i,j}
\end{multline*}

% done
\subsection{Аппроксимация на левой границе}
В качестве уравнения разностной схемы для $i = 0$ и $j = \overline{1,N_y-1}$ возьмем
известное граничное условие.
\[ u \MyVert{x = a} = g_1(y) \]
\[ v_{0,j} = g_1(y_j) \]

% done
\subsection{Аппроксимация на правой границе}
Проинтегрируем уравнение (\ref{main-equation}) на правой границе сетки.
\[
  - \left[
  \underbrace{ \Int{x_{N_x-1/2}}{x_{N_x}}\Int{y_{j -1/2}}{y_{j + 1/2}} \Inner{\pdv{x}\left( k_1 \pdv{u}{x} \right)} \Ddd{x}{y} }_{I_1} +
  \underbrace{ \Int{x_{N_x-1/2}}{x_{N_x}}\Int{y_{j -1/2}}{y_{j + 1/2}} \Inner{\pdv{y}\left( k_2 \pdv{u}{y} \right)} \Ddd{x}{y} }_{I_2}
  \right] =
  \underbrace{ \Int{x_{N_x-1/2}}{x_{N_x}}\Int{y_{j -1/2}}{y_{j + 1/2}} \Inner{f} \Ddd{x}{y} }_{I_3}
\]
Проинтегрируем левую часть один раз.
\[ I_1 = \Int{y_{j -1/2}}{y_{j + 1/2}} \Inner{k_1(x_{{N_x}},y) \pdv{u}{x} \MyVert{x = x_{{N_x}}}} \Dd{y} - \Int{y_{j - 1/2}}{y_{j + 1/2}} \Inner{k_1(x_{N_x-1/2},y) \pdv{u}{x} \MyVert{x = x_{N_x-1/2}}} \Dd{y} \]
\[ I_2 = \Int{x_{N_x-1/2}}{x_{{N_x}}} \Inner{k_2(x,y_{j + 1/2}) \pdv{u}{y} \MyVert{y = y_{j + 1/2}}} \Dd{x} - \Int{x_{N_x-1/2}}{x_{{N_x}}} \Inner{k_2(x,y_{j - 1/2}) \pdv{u}{y} \MyVert{y = y_{j -1/2}}} \Dd{x} \]
Найдем интегралы с помощью формулы правых прямоугольников.
\[ I_1 \approx h_y k_1(x_{{N_x}},y_{j}) \pdv{u}{x} \MyVvert{x = x_{{N_x}}}{y = y_{j}} - h_y k_1(x_{N_x-1/2},y_j) \pdv{u}{x} \MyVvert{x = x_{N_x-1/2}}{y = y_{j}} \]
\[ I_2 \approx \frac{h_x}{2} k_2(x_{N_x},y_{j+1/2}) \pdv{u}{y} \MyVvert{x = x_{{N_x}}}{y = y_{j + 1/2}} - \frac{h_x}{2} k_2(x_{N_x},y_{j-1/2}) \pdv{u}{y} \MyVvert{x = x_{{N_x}}}{y = y_{j-1/2}} \]
Теперь воспользуемся формулами численного дифференцирования и известным граничным условием.
\[ I_1 \approx h_y (- \chi_2 v_{{N_x},j} + g_2(y_j) ) - h_y k_1(x_{N_x-1/2},y_j) \frac{v_{{N_x},j} - v_{{N_x}-1,j}}{h_x} \]
\[ I_2 \approx \frac{h_x}{2} k_2(x_{N_x},y_{j+1/2}) \frac{v_{{N_x},j+1} - v_{{N_x},j}}{h_y} - \frac{h_x}{2} k_2(x_{N_x},y_{j-1/2}) \frac{v_{{N_x},j} - v_{{N_x},j-1}}{h_y} \]
Также с помощью формул средних и правых прямоугольников найдем интеграл в правой части.
\[ I_3 \approx \frac{h_x}{2} h_y f_{N_x,j} \]
Получим уравнение разностной схемы для $i = N_x$ и $j = \overline{1,N_y-1}$.
\begin{multline*}
  - \left[
    h_y (- \chi_2 v_{{N_x},j} + g_2(y_j) ) - h_y k_1(x_{N_x-1/2},y_j) \frac{v_{{N_x},j} - v_{{N_x}-1,j}}{h_x} + \right. \\
  \left. +
  \frac{h_x}{2} k_2(x_{N_x},y_{j+1/2}) \frac{v_{{N_x},j+1} - v_{{N_x},j}}{h_y} - \frac{h_x}{2} k_2(x_{N_x},y_{j-1/2}) \frac{v_{{N_x},j} - v_{{N_x},j-1}}{h_y}
  \right] =
  h_x \frac{h_x}{2} h_y f_{N_x,j}
\end{multline*}

% done
\subsection{Аппроксимация на нижней границе}
Проинтегрируем уравнение (\ref{main-equation}) на нижней границе сетки.
\[
  - \left[
  \underbrace{ \Int{x_{i-1/2}}{x_{i+1/2}}\Int{y_{0}}{y_{1/2}} \Inner{\pdv{x}\left( k_1 \pdv{u}{x} \right)} \Ddd{x}{y} }_{I_1} +
  \underbrace{ \Int{x_{i-1/2}}{x_{i+1/2}}\Int{y_{0}}{y_{1/2}} \Inner{\pdv{y}\left( k_2 \pdv{u}{y} \right)} \Ddd{x}{y} }_{I_2}
  \right] =
  \underbrace{ \Int{x_{i-1/2}}{x_{i+1/2}}\Int{y_{0}}{y_{1/2}} \Inner{f} \Ddd{x}{y} }_{I_3}
\]
Проинтегрируем левую часть один раз.
\[ I_1 = \Int{y_{0}}{y_{1/2}} \Inner{k_1(x_{i+1/2},y) \pdv{u}{x} \MyVert{x = x_{i+1/2}}} \Dd{y} - \Int{y_{0}}{y_{1/2}} \Inner{k_1(x_{i-1/2},y) \pdv{u}{x} \MyVert{x = x_{i-1/2}}} \Dd{y} \]
\[ I_2 = \Int{x_{i-1/2}}{x_{i+1/2}} \Inner{k_2(x,y_{1/2}) \pdv{u}{y} \MyVert{y = y_{1/2}}} \Dd{x} - \Int{x_{i-1/2}}{x_{i+1/2}} \Inner{k_2(x,y_{0}) \pdv{u}{y} \MyVert{y = y_{0}}} \Dd{x} \]
Найдем интегралы с помощью формулы левых прямоугольников.
\[ I_1 \approx \frac{h_y}{2} k_1(x_{i+1/2},y_0) \pdv{u}{x} \MyVvert{x = x_{i+1/2}}{y = y_{0}} - \frac{h_y}{2} k_1(x_{i-1/2},y_0) \pdv{u}{x} \MyVvert{x = x_{i-1/2}}{y = y_{0}} \]
\[ I_2 \approx h_x k_2(x_i,y_{1/2}) \pdv{u}{y} \MyVvert{x = x_{i}}{y = y_{1/2}} - h_x k_2(x_i,y_{0}) \pdv{u}{y} \MyVvert{x = x_{i}}{y = y_{0}} \]
Теперь воспользуемся формулами численного дифференцирования и известным граничным условием.
\[ I_1 \approx \frac{h_y}{2} k_1(x_{i+1/2},y_0) \frac{v_{i+1,0} - v_{i,0}}{h_x} - \frac{h_y}{2} k_1(x_{i-1/2},y_0) \frac{v_{i,0} - v_{i-1,0}}{h_x} \]
\[ I_2 \approx h_x k_2(x_i,y_{1/2}) \frac{v_{i,1} - v_{i,0}}{h_y} - h_x \left( \chi_3 v_{i,0} - g_3(x_i) \right) \]
Также с помощью формул средних и левых прямоугольников найдем интеграл в правой части.
\[ I_3 \approx h_x \frac{h_y}{2} f_{i,0} \]
Получим уравнение разностной схемы для $i = \overline{1,N_x-1}$ и $j = 0$.
\begin{multline*}
  - \left[
  \frac{h_y}{2} k_1(x_{i+1/2},y_0) \frac{v_{i+1,0} - v_{i,0}}{h_x} - \frac{h_y}{2} k_1(x_{i-1/2},y_0) \frac{v_{i,0} - v_{i-1,0}}{h_x} + \right. \\
  \left. +
  h_x k_2(x_i,y_{1/2}) \frac{v_{i,1} - v_{i,0}}{h_y} - h_x \left( \chi_3 v_{i,0} - g_3(x_i) \right)
  \right] =
  h_x \frac{h_y}{2} f_{i,0}
\end{multline*}

% done
\subsection{Аппроксимация на верхней границе}
Проинтегрируем уравнение (\ref{main-equation}) на верхней границе сетки.
\[
  - \left[
  \underbrace{ \Int{x_{i-1/2}}{x_{i+1/2}}\Int{y_{N_y-1/2}}{y_{N_y}} \Inner{\pdv{x}\left( k_1 \pdv{u}{x} \right)} \Ddd{x}{y} }_{I_1} +
  \underbrace{ \Int{x_{i-1/2}}{x_{i+1/2}}\Int{y_{N_y-1/2}}{y_{N_y}} \Inner{\pdv{y}\left( k_2 \pdv{u}{y} \right)} \Ddd{x}{y} }_{I_2}
  \right] =
  \underbrace{ \Int{x_{i-1/2}}{x_{i+1/2}}\Int{y_{N_y-1/2}}{y_{N_y}} \Inner{f} \Ddd{x}{y} }_{I_3}
\]
Проинтегрируем левую часть один раз.
\[ I_1 = \Int{y_{N_y-1/2}}{y_{N_y}} \Inner{k_1(x_{i+1/2},y) \pdv{u}{x} \MyVert{x = x_{i+1/2}}} \Dd{y} - \Int{y_{N_y-1/2}}{y_{N_y}} \Inner{k_1(x_{i-1/2},y) \pdv{u}{x} \MyVert{x = x_{i-1/2}}} \Dd{y} \]
\[ I_2 = \Int{x_{i-1/2}}{x_{i+1/2}} \Inner{k_2(x,y_{N_y}) \pdv{u}{y} \MyVert{y = y_{N_y}}} \Dd{x} - \Int{x_{i-1/2}}{x_{i+1/2}} \Inner{k_2(x,y_{N_y-1/2}) \pdv{u}{y} \MyVert{y = y_{N_y-1/2}}} \Dd{x} \]
Найдем интегралы с помощью формулы правых прямоугольников.
\[ I_1 \approx \frac{h_y}{2} k_1(x_{i+1/2},y_{N_y}) \pdv{u}{x} \MyVvert{x = x_{i+1/2}}{y = y_{N_y}} - \frac{h_y}{2} k_1(x_{i-1/2},y_{N_y}) \pdv{u}{x} \MyVvert{x = x_{i-1/2}}{y = y_{N_y}} \]
\[ I_2 \approx h_x k_2(x_i,y_{N_y}) \pdv{u}{y} \MyVvert{x = x_{i}}{y = y_{N_y}} - h_x k_2(x_i,y_{N_y-1/2}) \pdv{u}{y} \MyVvert{x = x_{i}}{y = y_{N_y-1/2}} \]
Теперь воспользуемся формулами численного дифференцирования и известным граничным условием.
\[ I_1 \approx \frac{h_y}{2} k_1(x_{i+1/2},y_{N_y}) \frac{v_{i+1,N_y} - v_{i,N_y}}{h_x} - \frac{h_y}{2} k_1(x_{i-1/2},y_{N_y}) \frac{v_{i,N_y} - v_{i-1,N_y}}{h_x} \]
\[ I_2 \approx h_x \left( - \chi_4 v_{i,N_y} + g_4(x_i) \right) - h_x k_2(x_i,y_{N_y-1/2}) \frac{v_{i,N_y} - v_{i,N_y-1}}{h_y} \]
Также с помощью формул средних и правых прямоугольников найдем интеграл в правой части.
\[ I_3 \approx h_x \frac{h_y}{2} f_{i,N_y} \]
Получим уравнение разностной схемы для $i = \overline{1,N_x-1}$ и $j = N_y$.
\begin{multline*}
  - \left[
  \frac{h_y}{2} k_1(x_{i+1/2},y_{N_y}) \frac{v_{i+1,N_y} - v_{i,N_y}}{h_x} - \frac{h_y}{2} k_1(x_{i-1/2},y_{N_y}) \frac{v_{i,N_y} - v_{i-1,N_y}}{h_x} + \right. \\
  \left. +
  h_x \left( - \chi_4 v_{i,N_y} + g_4(x_i) \right) - h_x k_2(x_i,y_{N_y-1/2}) \frac{v_{i,N_y} - v_{i,N_y-1}}{h_y}
  \right] =
  h_x \frac{h_y}{2} f_{i,N_y}
\end{multline*}

% done
\subsection{Аппроксимация в левой нижней граничной точке}
В качестве уравнения разностной схемы для $i = 0$ и $j = 0$ возьмем
известное граничное условие.
\[ u \MyVert{x = a} = g_1(y) \]
\[ v_{0,0} = g_1(y_j) \]

% done
\subsection{Аппроксимация в правой нижней граничной точке}
\[
  - \left[
  \underbrace{ \Int{x_{N_x - 1/2}}{x_{N_x}}\Int{y_{0}}{y_{1/2}} \Inner{\pdv{x}\left( k_1 \pdv{u}{x} \right)} \Ddd{x}{y} }_{I_1} +
  \underbrace{ \Int{x_{N_x - 1/2}}{x_{N_x}}\Int{y_{0}}{y_{1/2}} \Inner{\pdv{y}\left( k_2 \pdv{u}{y} \right)} \Ddd{x}{y} }_{I_2}
  \right] =
  \underbrace{ \Int{x_{N_x - 1/2}}{x_{N_x}}\Int{y_{0}}{y_{1/2}} \Inner{f} \Ddd{x}{y} }_{I_3}
\]
Проинтегрируем левую часть один раз.
\[ I_1 = \Int{y_{0}}{y_{1/2}} \Inner{k_1(x_{N_x},y) \pdv{u}{x} \MyVert{x = N_x}} \Dd{y} - \Int{y_{0}}{y_{1/2}} \Inner{k_1(x_{N_x-1/2},y) \pdv{u}{x} \MyVert{x = x_{N_x-1/2}}} \Dd{y} \]
\[ I_2 = \Int{x_{N_x - 1/2}}{x_{N_x}} \Inner{k_2(x,y_{1/2}) \pdv{u}{y} \MyVert{y = y_{1/2}}} \Dd{x} - \Int{x_{N_x - 1/2}}{x_{N_x}} \Inner{k_2(x,y_{0}) \pdv{u}{y} \MyVert{y = y_{0}}} \Dd{x} \]

Найдем интегралы с помощью формулы левых и правых прямоугольников.
\[ I_1 \approx \frac{h_y}{2} k_1(x_{N_x},y_{0}) \pdv{u}{x} \MyVvert{x = x_{N_x}}{y = y_{0}} - \frac{h_y}{2} k_1(x_{N_x-1/2},y_{0}) \pdv{u}{x} \MyVvert{x = x_{N_x-1/2}}{y = y_{0}} \]
\[ I_2 \approx \frac{h_x}{2} k_2(x_{N_x},y_{1/2}) \pdv{u}{y} \MyVvert{x = x_{N_x}}{y = y_{1/2}} - \frac{h_x}{2} k_2(x_{N_x},y_{0}) \pdv{u}{y} \MyVvert{x = x_{N_x}}{y = y_{0}} \]
Теперь воспользуемся формулами численного дифференцирования и известными граничным условием.
\[ I_1 \approx \frac{h_y}{2} (- \chi_2 v_{N_x, 0} + g_2(y_{0})) - \frac{h_y}{2} k1(x_{N_x -1}, y_{0}) \frac{v_{N_x,0} - v_{N_x-1,0}}{h_x} \]
\[ I_2 \approx \frac{h_x}{2} k_2(x_{N_x},y_{1/2}) \frac{v_{N_x,1} - v_{N_x,0}}{h_y} - \frac{h_x}{2} (\chi_3 v_{N_x, 0} + g_3(x_{N_x}))\]
Также с помощью формул левых и правых прямоугольников найдем интеграл в правой части.
\[ I_3 \approx \frac{h_x}{2} \frac{h_y}{2} f_{N_x, 0} \]
Получим уравнение разностной схемы для $i = N_x$ и $j = 0$.
\begin{multline*}
  - \left[
    \frac{h_y}{2} (- \chi_2 v_{N_x, 0} + g_2(y_{0})) - \frac{h_y}{2} k1(x_{N_x -1}, y_{0}) \frac{v_{N_x,0} - v_{N_x-1,0}}{h_x} + \right. \\
  \left. +
  \frac{h_x}{2} k_2(x_{N_x},y_{1/2}) \frac{v_{N_x,1} - v_{N_x,0}}{h_y} - \frac{h_x}{2} (\chi_3 v_{N_x, 0} + g_3(x_{N_x}))
  \right] =
  \frac{h_x}{2} \frac{h_y}{2} f_{N_x, 0} 
\end{multline*}

% done
\subsection{Аппроксимация в правой верхней граничной точке}
Проинтегрируем уравнение (\ref{main-equation}) в правой верхней граничной точке.
\[
  - \left[
  \underbrace{ \Int{x_{N_x - 1/2}}{N_x}\Int{y_{N_y-1/2}}{y_{N_y}} \Inner{\pdv{x}\left( k_1 \pdv{u}{x} \right)} \Ddd{x}{y} }_{I_1} +
  \underbrace{ \Int{x_{N_x - 1/2}}{N_x}\Int{y_{N_y-1/2}}{y_{N_y}} \Inner{\pdv{y}\left( k_2 \pdv{u}{y} \right)} \Ddd{x}{y} }_{I_2}
  \right] =
  \underbrace{ \Int{x_{N_x - 1/2}}{N_x}\Int{y_{N_y-1/2}}{y_{N_y}} \Inner{f} \Ddd{x}{y} }_{I_3}
\]
Проинтегрируем левую часть один раз.
\[ I_1 = \Int{y_{N_y-1/2}}{y_{N_y}} \Inner{k_1(x_{N_x},y) \pdv{u}{x} \MyVert{x = N_x}} \Dd{y} - \Int{y_{N_y-1/2}}{y_{N_y}} \Inner{k_1(x_{N_x-1/2},y) \pdv{u}{x} \MyVert{x = x_{N_x-1/2}}} \Dd{y} \]
\[ I_2 = \Int{x_{N_x-1/2}}{x_{N_x}} \Inner{k_2(x,y_{N_y}) \pdv{u}{y} \MyVert{y = y_{N_y}}} \Dd{x} - \Int{N_x - 1/2}{N_x} \Inner{k_2(x,y_{N_y-1/2}) \pdv{u}{y} \MyVert{y = y_{N_y-1/2}}} \Dd{x} \]
Найдем интегралы с помощью формулы правых прямоугольников.
\[ I_1 \approx \frac{h_y}{2} k_1(x_{N_x},y_{N_y}) \pdv{u}{x} \MyVvert{x = x_{N_x}}{y = y_{N_y}} - \frac{h_y}{2} k_1(x_{N_x-1/2},y_{N_y}) \pdv{u}{x} \MyVvert{x = x_{N_x-1/2}}{y = y_{N_y}} \]
\[ I_2 \approx \frac{h_x}{2} k_2(x_{N_x},y_{N_y}) \pdv{u}{y} \MyVvert{x = x_{N_x}}{y = y_{N_y}} - \frac{h_x}{2} k_2(x_{N_x},y_{N_y-1/2}) \pdv{u}{y} \MyVvert{x = x_{N_x}}{y = y_{N_y-1/2}} \]
Теперь воспользуемся формулами численного дифференцирования и известными граничным условием.
\[ I_1 \approx \frac{h_y}{2} (- \chi_2 v_{{N_x},N_y} + g_2(y_{N_y}) ) - \frac{h_y}{2} k_1(x_{N_x-1/2},y_{N_y}) \frac{v_{{N_x},N_y} - v_{{N_x}-1,N_y}}{h_x} \]
\[ I_2 \approx \frac{h_x}{2} \left( - \chi_4 v_{N_x,N_y} + g_4(x_{N_x}) \right) - \frac{h_x}{2} k_2(x_{N_x},y_{N_y-1/2}) \frac{v_{N_x,N_y} - v_{N_x,N_y-1}}{h_y} \]
Также с помощью формул средних и правых прямоугольников найдем интеграл в правой части.
\[ I_3 \approx \frac{h_x}{2} \frac{h_y}{2} f_{N_x, N_y} \]
Получим уравнение разностной схемы для $i = N_x$ и $j = N_y$.
\begin{multline*}
  - \left[
    \frac{h_y}{2} (- \chi_2 v_{{N_x},N_y} + g_2(y_{N_y}) ) - \frac{h_y}{2} k_1(x_{N_x-1/2},y_{N_y}) \frac{v_{{N_x},N_y} - v_{{N_x}-1,N_y}}{h_x} + \right. \\
  \left. +
  \frac{h_x}{2} \left( - \chi_4 v_{N_x,N_y} + g_4(x_{N_x}) \right) - \frac{h_x}{2} k_2(x_{N_x},y_{N_y-1/2}) \frac{v_{N_x,N_y} - v_{N_x,N_y-1}}{h_y}
  \right] =
  \frac{h_x}{2} \frac{h_y}{2} f_{N_x, N_y}
\end{multline*}

% done
\subsection{Аппроксимация в левой верхней граничной точке}
В качестве уравнения разностной схемы для $i = 0$ и $j = N_y$ возьмем
известное граничное условие.
\[ u \MyVert{x = a} = g_1(y) \]
\[ v_{0,N_y} = g_1(y_j) \]

Полученную разностную схему будем использовать для решения исходного ДУ.