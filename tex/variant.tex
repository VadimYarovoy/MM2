\textbf{Вариант В12.} Используя интегро-интерполяционный метод,
разработать программу для моделирования распределения температуры
в брусе, описываемого математической моделью
\begin{equation}\label{main-equation}
  - \left[
    \pdv{x} \left( k_1(x, y) \pdv{u}{x} \right) +
    \pdv{y} \left( k_2(x, y) \pdv{u}{y} \right)
  \right] = f(x, y)
\end{equation}
\[ a \leq x \leq b \]
\[ c \leq y \leq d \]
\[ 0 < c_{11} \leq k_1 \leq c_{12} \]
\[ 0 < c_{21} \leq k_2 \leq c_{22} \]
с граничными условиями
\begin{align*}
  u \MyVert{x = a} &= g_1(y)
  &
  - k_1 \pdv{u}{x}\MyVert{x = b} &= \chi_2 u \MyVert{x = b} - g_2(y) \\
  k_2 \pdv{u}{y}\MyVert{y = c} &= \chi_3 u \MyVert{y = c} - g_3(x)
  &
  - k_2 \pdv{u}{y}\MyVert{y = d} &= \chi_4 u \MyVert{y = d} - g_4(x)
\end{align*}
% Для решения системы линейных алгебраических уравнений использовать
% метод сопряженных градиентов с предобусловливанием. Матрица
% алгебраической системы должна храниться в упакованной форме -- сжатый
% разреженный строчный вид (4).
